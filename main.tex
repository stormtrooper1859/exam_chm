\documentclass[letterpaper,12pt]{article}
\usepackage{tabularx} % extra features for tabular environment
\usepackage{amsmath}  % improve math presentation
\usepackage{graphicx} % takes care of graphic including machinery
\usepackage[margin=1in,letterpaper]{geometry} % decreases margins
\usepackage{cite} % takes care of citations
\usepackage[final]{hyperref} % adds hyper links inside the generated pdf file
\usepackage{subfiles}
\graphicspath{{images/}{../images/}}
\usepackage[T2A]{fontenc}
\usepackage[utf8]{inputenc}
\usepackage[russian]{babel}

\hypersetup{
	colorlinks=true,       % false: boxed links; true: colored links
	linkcolor=blue,        % color of internal links
	citecolor=blue,        % color of links to bibliography
	filecolor=magenta,     % color of file links
	urlcolor=blue         
}
\usepackage{blindtext}
%++++++++++++++++++++++++++++++++++++++++


\begin{document}

\title{Билет 11, Интерполяция сплайнами}
\date{\today}
\maketitle

\subsubsection*{Команда}
$\begin{array}{lr}
\textup{Устинов А.П.}\ &\ \textup{M}3336 \\
\textup{Акифьев Д.А.}\ &\ \textup{M}3337 \\
\textup{Акназаров А.Р.}\ &\ \textup{M}3336 \\
\textup{Мещеряков Н.Д.}\ &\ \textup{M}3335 \\
\textup{Осипов А.А.}\ &\ \textup{M}3336 
\end{array}$

\begin{abstract}
    Задача интерполяции - по заданному набору значений функции $f(x)$ на сетке $\{x_i\}_{i=0}^N$ построить функцию $U(x)$, совпадающую с $f(x_i)$ в узлах $x_i$
\end{abstract}

\section{Интерполяция сплайнами}

\subfile{sections/intro}

\section{Сплайны третьего порядка}

\subfile{sections/spline3_intro}


\subsection{Локальная интерполяция}

\subfile{sections/local_interpol}
	
\subsection{Глобальная интерполяция}

\subfile{sections/global_interpol}

\subsubsection*{Сведение матрицы к трехдиагональной}

\subfile{sections/tridiagonal}

\section{Физическая интерпретация}

\subfile{sections/phys}

\end{document}
