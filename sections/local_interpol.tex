Первым способом будет явно задать для каждого $x_i$ значения первой производной $s_i$ (которые, как замечено ранее, равны $b_i$) интерполяционных полиномов $P_{3,i}$, которые достигаются на концах интервалов $[x_{i-1}, x_i]$.
\begin{equation}
    P_{3,i}' (x_{i-1}) = s_{i-1} = b_{i-1},\  P_{3,i}' (x_i) = s_i = b_i,\ i = 1 \dots n			\label{Eq:local_interpol}
\end{equation}
Из \eqref{Eq:local_interpol} выводится условие непрерывности первой производной функции $S_3$:
\begin{equation*}
    P_{3,i}' (x_{i-1}) = P_{3,i+1}' (x_i) = s_i = b_i,\ i = 1\dots n-1
\end{equation*}
Решая систему уравнений \eqref{Eq:spline_3} и \eqref{Eq:local_interpol}, мы находим полиномы $P_{3,i}$
\begin{equation}
\begin{split}
    P_{3,i} (x) & =
    \frac{(x-x_i)^2 [2 (x-x_{i-1}) + (x_i-x_{i-1})]} {(x_i-x_{i-1})^3}
    \cdot y_{i-1} \\
    & + \frac{(x-x_{i-1} )^2 [2(x_i-x)+(x_i-x_{i-1} )]} {(x_i-x_{i-1} )^3}
    \cdot y_i \\
	& + \frac {(x-x_i )^2 (x-x_{i-1} )}{(x_i-x_{i-1} )^2} 
	\cdot  s_{i-1} \\
	& +	\frac{(x-x_{i-1} )^2 (x-x_i )} {(x_i-x_{i-1} )^2}
	\cdot s_i
\end{split}
\end{equation}
	Построенный сплайн имеет степень гладкости $p=1$ и дефект $n-p=2$.