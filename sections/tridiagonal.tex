Систему \eqref{Sys:step0} при наложении краевых условий можно значительно упростить.

Положим, что $S_3''(x_0) = S_3''(x_n) = c_0 = c_n = 0$, а также избавимся от $a_i$, введя так называемые разделенные разности Ньютона:
$$f(x_{i-1}, x_i) = \frac{f(x_i) - f(x_{i-1})}{x_i - x_{i-1}} = \frac{a_i - a_{i-1}}{h_i}$$

\begingroup
\Large
\begin{equation}
\begin{array}{lr}
     b_i - \frac{c_i}{2}h_i + \frac{d_i}{6}h_i^2= f(x_{i-1}, x_i)\ \ \ \  & i = 2,\dots, n  \\
     b_{i-1} = b_i - c_i h_i +\frac{d_i}{2}h_i^2 & i = 2,\dots, n  \\
     c_{i-1} = c_i - d_i h_i & i = 2,\dots, n  \\
     b_1 - \frac{c_1}{2} h_1 + \frac{d_1}{6} h_1^2 = f(x_0,x_1)
     & \\
     c_n = 0 & \\
     c_1 - d_1 h_1 = 0 &
\end{array}
\label{Sys:step1}
\end{equation}
\endgroup

Выразим из \eqref{Sys:step1} $d_1 h_1 = c_1$, и $d_i h_i = c_i - c_{i -1}$

\begingroup
\Large
\begin{equation}
\begin{array}{lr}
     b_i - \frac{c_i}{2}h_i + \frac{h_i}{6}(c_i - c_{i-1})= f(x_{i-1}, x_i)\ \ \ \  & i = 2,\dots, n  \\
     b_{i-1} = b_i - c_i h_i +\frac{h_i}{2}(c_i - c_{i-1}) & i = 2,\dots, n  \\
     b_1 - \frac{c_1}{2} h_1 + \frac{h_1}{6} c_1 = f(x_0,x_1)
     & \\
     c_n = 0 & \\
\end{array}
\label{Sys:step2}
\end{equation}
\endgroup

Приведем подобные коэффициенты при $c_i$ и выразим $b_i$: $$b_1 = \frac{c_1 h_1}{3} + f(x_0,x_1)$$ и $$b_i = \frac{c_i h_i}{3} + \frac{c_{i-1}}{6} h_i + f(x_{i-1}, x_i)$$ и подставим их в систему \eqref{Sys:step2}, формально доопределив $c_0 = 0$ из условий естественного сплайна 

\begin{equation}
\begin{array}{lr}
    \frac{h_{i-1}}{6}c_{i-2} + (\frac{h_i}{3} + \frac{h_{i-1}}{3}) c_{i-1} + \frac{h_i}{6}c_i = f(x_{i-1},x_i) - f(x_{i-2}, x_{i-1}),\ \ \ & i = 2,\dots,n \\
    & \\
    c_0 = c_n = 0 &
\end{array}
\label{Sys:step3}
\end{equation}

В результате серии упрощений мы получили систему относительно только $c_i$, причем матрица такой системы имеет трехдиагональный вид, т.е. ненулевыми являются элементы только главной диагонали и двух соседних. Такие системы решаются легко решаются методом прогонки.

\begingroup
\Large
\begin{equation*}
\begin{array}{cccccccr}
2c_{1} & + & \frac{h_2}{h_1+h_2} c_2 & & & = & 6f(x_{i-2}, x_{i-1}, x_i) \\
 & \ddots & & & & & & \\
\frac{h_i} {h_i + h_{i+1}} & + & 2 c_i & + & \frac{h_{i+1}}{h_i + h_{i+1}} c_{i+1} & = & 6f(x_{i-1}, x_i, x_{i+1}) \\
 & & & \ddots & & & & \\
 & & \frac{h_{n-1}}{h_{n-1} + h_{n}} c_{n-2} & + & 2c_{n-1} & = & 6f(x_{n-2}, x_{n-1}, x_n)
\end{array}
\end{equation*}
\endgroup

Следует отметить, что перед приведением к матрице мы домножили \eqref{Sys:step3} на $\frac{6} {h_i +h_{i-1}}$, и использовали разделенные разности следуещего третьего порядка $$f(x_{i-2},x_{i-1},x_i) = \frac{f(x_{i-1}, x_i) - f(x_{i-2}, x_{i-1})}{x_i - x_{i-2}}= \frac{f(x_{i-1}, x_i) - f(x_{i-2}, x_{i-1})}{h_i + h_{i-1}}$$

Решив такую систему уравнений мы получим сплайн со степенью $3$, степень. гладкости $2$, и дефектом $1$